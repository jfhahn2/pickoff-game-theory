

\documentclass{article}

\usepackage{amsfonts, amsmath, graphicx}
\usepackage[margin=1in]{geometry}
\usepackage{natbib}
\bibliographystyle{apa-good}

\title{The Two-Foot Rule: A game theoretic analysis of the pickoff limit in Major League Baseball}

\begin{document}

  \maketitle

  \section{Introduction}

    In baseball, the stolen base is an exciting play in which a runner will take off on the pitch, attempting to reach the next base before the catcher can throw the ball to the base for a fielder to tag the runner out. Major League Baseball (MLB) stolen base totals peaked in the 1980s and 1990s but then waned in the 2000s and 2010s as teams cut back on the play, which generally requires a high success rate in order to have a positive impact on run scoring \citep{tango_book_2007}. On the micro level, it improves a team's win probability to reduce stolen base attempts that have negative expected outcomes, but on the macro level, the reduction in stolen base attempts affects the entertainment value of the game.

    Acknowledging this and other trends, MLB introduced several rule changes taking effect in the 2023 regular season, with the goal of making the game more exciting \citep{castrovince_pitch_2023}. With these changes, a pitch clock limits the time between pitches, and there are restrictions on infielder positioning, curtailing defensive shifts. Two changes are particularly relevant to stolen base attempts: Bases are now wider (reducing the distance between first and second base by 4.5 inches), and there are now count limitations on pickoff attempts, a previously unlimited deterrent of stolen bases.
    
    Whether attempting a stolen base or not, runners typically take a {\it lead} from their base, commonly positioning themselves 8--12 feet away from their base, in the direction of the next base. The longer the lead, the shorter the distance need to run on a stolen base attempt. On a pickoff attempt, instead of throwing a pitch, the pitcher will disengage from the pitching rubber and throw the ball behind the runner to a fielder covering the base, attempting to tag the runner out. Under the new rules, if the pitcher disengages from the pitching rubber three times without successfully picking off the runner, then the runner automatically advances to the next base.

    This disengagement limit creates an interesting cat-and-mouse game between the pitcher and the runner. Each unsuccessful pickoff attempt disincentivizes the pitcher from attempting another pickoff, which incentivizes the runner to increase their lead distance. But how much should the runner extend their lead? And how does the answer change depending on runner speed and other context?
    
    In the present work, we conduct the first public analysis of this game. We model the baseball inning as a Markov decision process and solve for (a) the optimal policy for runner lead distance given current pitcher pickoff behavior and (b) the game theoretic equilibrium behavior in the two-player game between runner and pitcher. We present the practically actionable rule-of-thumb that an average runner should increase their lead distance by two feet after each disengagement.

    \subsection{Related work}

    \cite{downey_pick_2015} previously analyzed pickoff throws and stolen base attempts as a mixed strategy game without using lead distance data. They modeled the two-player game as a simultaneous decision in which the pitcher chooses whether to attempt a pickoff and the runner chooses whether to attempt a stolen base, showing that this theoretical game has a mixed-strategy Nash equilibrium. Under this equilibrium, left-handed pitchers (possessing better pickoff moves than their right-handed counterparts) would attempt fewer pickoffs and see fewer stolen bases attempted against them, a result validated by empirical data. In a later analysis, \cite{downey_pressure_2019} argued that pitchers can effectively implement a mixed strategy in low-pressure situations but struggle to randomize their decisions as pressure increases, showing a higher auto-correlation for sequential pickoff decisions in high-pressure situations. Both of these analyses covered a game in which pickoff attempts were unlimited.

    Several studies in sports economics literature have examined the extent to which professional athletes play minimax strategy. \cite{palacios-huerta_professionals_2003} found evidence consistent with minimax equilibrium behavior in the mixed strategy game of the penalty kick, in which the kicker chooses to shoot the ball left or right and the goalkeeper simultaneously chooses to dive left or right for the save. \cite{palacios-huerta_experientia_2008} followed this with a laboratory experiment featuring professional soccer players and college students simulating the penalty kick decision using cards, showing that the professionals played minimax but the students did not. On the other hand, \cite{kovash_professionals_2009} found evidence inconsistent with minimax equilibrium behavior in the National Football League (teams called more run plays and fewer pass plays than theoretically optimal) and in MLB (pitchers threw more fastballs than theoretically optimal).

    The Markov process is a very popular model in sports analytics, particularly in baseball, where the game's highly structured progression lends it well to a state space model \citep{bukiet_markov_1997}. \cite{hirotsu_using_2002} used a four-state Markov model for soccer to inform substitution decision-making based on how substitutions impact transition probabilities between states. \cite{hirotsu_markov_2003} used similar methodology to inform substitution (pinch hitting) decision-making in baseball. Recently, \cite{chan_points_2021} used a Markov model to evaluate team performance in football. More specifically, the Markov decision process has become particularly popular model in recent years for analyzing decision-making in sports, with applications in tennis \citep{nadimpalli_when_2013}, baseball \citep{hirotsu_using_2019}, basketball \citep{sandholtz_markov_2020}, darts \citep{baird_optimising_2020} and football \citep{biro_reinforcement_2022}.

    An intermediate result of the present work is a probability model for stolen base attempts and successes (as well as an analysis how those probabilities changed with the 2023 MLB rule changes), a topic which has not received much attention. \cite{loughin_assessing_2008} estimated linear mixed-effects models for stolen base attempts and successes, using random effects for pitcher and catcher (not the runner, notably) and fixed effects for balls, strikes, outs, score, venue and pitcher hand. They showed that the spread of pitcher effects is slightly greater than the spread of catcher effects. That analysis did not have data on runner sprint speed and catcher arm strength. More recently, \cite{stanley_modeling_2023} modeled stolen base success probability using both logistic regression and a random forest, including runner sprint speed and catcher arm strength as features but excluding player identities. That analysis was limited to data prior to the 2023 MLB rule changes that incentivized base stealing.

  \section{Data}

    \subsection{MLB Stats API}

    \subsection{Baseball Savant}

    \subsection{Proprietary Statcast Data}

  \section{Methods}

    \subsection{Notation}

    Throughout this section, we use the following notation to describe the data observed at the beginning of each play $i \in \{1, ..., n\}$ (which can be a pitch or a pickoff attempt):
    \begin{align*}
      b_i & \in \{0, 1\}^3 \mbox{ denotes which bases (1B, 2B, 3B, respectively) are occupied;}\\
      (c_i^B, c_i^S) \equiv c_i & \in \{0, 1, 2, 3\} \times \{0, 1, 2\} \mbox{ denotes the ball-strike count;}\\
      d_i & \in \{0, 1, 2\} \mbox{ denotes the number of disengagements already used; and}\\
      o_i & \in \{0, 1, 2\} \mbox{ denotes the number of outs.}
    \end{align*}
    We use $\mathcal{H}^R$, $\mathcal{H}^P$, and $\mathcal{H}^C$ to denote the sets of runners, pitchers and catchers, respectively. On play $i$:
    \begin{align*}
      h_i^R & \in \mathcal{H}^R \mbox{ is the runner;}\\
      h_i^P & \in \mathcal{H}^P \mbox{ is the pitcher;}\\
      h_i^C & \in \mathcal{H}^C \mbox{ is the catcher;}\\
      z_i^R & \in \mathbb{R^+} \mbox{ is the sprint speed of runner $h_i^R$; and}\\
      z_i^C & \in \mathbb{R^+} \mbox{ is the arm strength of catcher $h_i^C$.}
    \end{align*}
    When the play begins, we use the following notation to describe the actions taken by the players:
    \begin{align}
    \label{eqn:runner-outcome}
      \begin{split}
        \ell_i  \in &~ \mathbb{R}^+ \mbox{ denotes the lead distance (in feet) taken by the first-base runner;}\\
        p_i \in &~ \{0, 1\} \mbox{ denotes whether the pitcher attempts (1) or does not attempt (0) a pickoff; and}\\
        r_i \in &~ \{\mbox{P}^+,\, \mbox{P}^-,\, \mbox{S}^+,\, \mbox{S}^-,\, \mbox{N}\} \equiv \mathcal{R} \mbox{ denotes the runner outcome, where}\\
          & \mbox{P}^+  \mbox{ represents a successful pickoff attempt (runner is out);}\\
          & \mbox{P}^-  \mbox{ represents an unsuccessful pickoff attempt (runner is safe);}\\
          & \mbox{S}^+  \mbox{ represents a successful stolen base attempt (runner is safe);}\\
          & \mbox{S}^-  \mbox{ represents an usuccessful stolen base attempt (runner is out); and}\\
          & \mbox{N}~   \mbox{ represents no runner action (a pitch with no stolen base attempt).}
%        p_i & \in \{0, 1\} \mbox{ indicates whether the pitcher attempts a pickoff on the first-base runner; and}\\
%        s_i & \in \{0, 1\} \mbox{ indicates whether the first-base runner attempts a stolen base.}
      \end{split}
    \end{align}
    If there is no runner on first base, then $\ell_i$ and $r_i$ are undefined.% We use a superscript to denote the success/failure of pickoff and stolen base attempts: 
%    \begin{align*}
%      p_i^+ & \in \{0, 1\} \mbox{ denotes whether the pickoff attempt is successful (i.e. the runner is tagged out); and}\\
%      s_i^+ & \in \{0, 1\} \mbox{ denotes whether the stolen base attempt is successful.}
%    \end{align*}

    \subsection{Probability model for runner outcomes}
    \label{sec:prob-runner-outcome}

      As itemized in (\ref{eqn:runner-outcome}), there are five possible outcomes for the runner with respect to the run game. To estimate each of these probabilities conditional on player identities, count and lead distance, we model four probabilities. Using $R_i$ to denote the random variable governing the probability distribution over $r_i$,
      \begin{align*}
        \pi_i    &= \mathbb{P}(R_i \in \{\mbox{P}^+,\, \mbox{P}^-\}) \mbox{ is the pickoff attempt probability;}\\
        \pi_i^+  &= \mathbb{P}(R_i \in \{\mbox{P}^+\} \mid R_i \in \{\mbox{P}^+,\, \mbox{P}^-\}) \mbox{ is the pickoff success probability;}\\
        \psi_i   &= \mathbb{P}(R_i \in \{\mbox{S}^+,\, \mbox{S}^-\} \mid R_i \in \{\mbox{S}^+,\, \mbox{S}^-, N\}) \mbox{ is the stolen base attempt probability; and}\\
        \psi_i^+ &= \mathbb{P}(R_i \in \{\mbox{S}^+\} \mid R_i \in \{\mbox{S}^+,\, \mbox{S}^-\}) \mbox{ is the stolen base success probability.}
      \end{align*}
      We estimate each of these probabilities with generalized linear mixed-effects models, implemented in R using the package lme4 \citep{bates_fitting_2015}.

      \subsubsection{Pickoff attempt probability}
      \label{sec:prob-po-attempt}

        We model $\pi_i$ using a mixed-effects logistic regression with fixed effects for balls, strikes, outs, disengagements (as a categorical), and lead distance; and with a random effect for the pitcher.
        \begin{align}
          \label{eqn:prob-po-attempt}
          \begin{split}
            \log\left(\frac{\pi_i}{1 - \pi_i}\right) &= \alpha + \beta^B c_i^B + \beta^S c_i^S + \beta^O o_i + \beta^D_{d_i} + \beta^L \ell_i + \gamma^P_{h_i^P}\\
            \gamma^P_{h} &\sim \mathcal{N}(0, \sigma^2_P) \hspace{4mm} \mbox{\it i.i.d.} \hspace{4mm} \forall h \in \mathcal{H}^P.
          \end{split}
        \end{align}
        This model has eight fixed, unknown parameters: the intercept $\alpha$; the five slopes $\beta^B$, $\beta^S$, $\beta^O$, $\beta^D$ and $\beta^L$; and the two variance parameters $\sigma^2_R$ and $\sigma^2_P$. The number of random effects is $|\mathcal{H}^R| + |\mathcal{H}^P|$.
        
      \subsubsection{Pickoff success probability}
      \label{sec:prob-po-success}

        We model $\pi_i^+$ using a mixed-effects logistic regression with a fixed effect for lead distance and a random effect for pitcher. These were the only variables for which we found substantive effects.
        \begin{align}
          \label{eqn:prob-po-success}
          \begin{split}
            \log\left(\frac{\pi_i^+}{1 - \pi_i^+}\right) &= \alpha + \beta^L \ell_i + \gamma^P_{h_i^P},\\
            \gamma^P_{h} &\sim \mathcal{N}(0, \sigma^2_P) \hspace{4mm} \mbox{\it i.i.d.} \hspace{4mm} \forall h \in \mathcal{H}^P.
          \end{split}
        \end{align}
        This model has three fixed, unknown parameters: the intercept $\alpha$, the slope $\beta^L$, and the variance parameters $\sigma^2_P$. The number of random effects is $|\mathcal{H}^P|$.

      \subsubsection{Stolen base attempt probability}
      \label{sec:prob-so-attempt}

        Modeling the runner's decision to attempt a stolen base is complicated. One may think of this as a deterministic decision made by the runner based on the expected run value with and without attempting a steal. Such a model aligns very poorly with real decisions made by runners. We posit that the runner's ability to attempt a stolen base is influenced by external factors such as the runner's perception of the pitcher's body language timing between pitches. For this reason, we model stolen base attempts as a stochastic process rather than a deterministic decision.

        Intentionally, we exclude lead distance from the stolen base attempt model. While lead distance does help explain the probability of a stolen base attempt, this can cause problems with the framing of the runner's decision. When including lead distance, we find situations in which the expected value of a stolen base attempt is negative, and the optimal lead distance is zero feet to minimize the probability of a stolen base attempt. This is a poor reflection of reality because the runner always has the power to choose not to attempt a stolen base. We obtain a more fit-for-purpose model by excluding lead distance.

        We model $\psi_i$ using a mixed-effects logistic regression with fixed effects for balls, strikes, outs, disengagements (as a categorical), runner sprint speed, and catcher arm strength; and with random effects for runner, pitcher and catcher.
        \begin{align}
          \label{eqn:prob-sb-attempt}
          \begin{split}
            \log\left(\frac{\psi_i}{1 - \psi_i}\right) &= \alpha + \beta^B c_i^B + \beta^S c_i^S + \beta^O o_i + \beta^D_{d_i} + (\beta^R z_i^R + \gamma^R_{h_i^R}) + \gamma^P_{h_i^P} + (\beta^C z_i^C + \gamma^C_{h_i^C}),\\
            \gamma^R_{h} &\sim \mathcal{N}(0, \sigma^2_R) \hspace{4mm} \mbox{\it i.i.d.} \hspace{4mm} \forall h \in \mathcal{H}^R,\\
            \gamma^P_{h} &\sim \mathcal{N}(0, \sigma^2_P) \hspace{4mm} \mbox{\it i.i.d.} \hspace{4mm} \forall h \in \mathcal{H}^P,\\
            \gamma^C_{h} &\sim \mathcal{N}(0, \sigma^2_C) \hspace{4mm} \mbox{\it i.i.d.} \hspace{4mm} \forall h \in \mathcal{H}^C.
          \end{split}
        \end{align}
 
        We interpret $(\beta^R z_i^R + \gamma^R_{h_i^R})$ as the effect of runner $h_i^R$ on the attempt probability, combining the fixed effect of the runner's sprint speed with the random effect of their identity. By including a fixed effect for sprint speed, we are effectively regularizing the runner effects toward priors based on their sprint speeds. The interpretation of $(\beta^C z_i^C + \gamma^C_{h_i^C})$ for catchers is similar: Each catcher has an estimated effect regularized toward a prior based on their arm strength.
      
      \subsubsection{Stolen base success probability}
      \label{sec:prob-so-success}

        We model $\psi_i^+$ using a mixed-effects logistic regression with fixed effects for lead distance, runner sprint speed, and catcher arm strength; and with random effects for runner, pitcher and catcher.
        \begin{align}
          \label{eqn:prob-sb-success}
          \begin{split}
            \log\left(\frac{\psi_i^+}{1 - \psi_i^+}\right) &= \alpha + \beta^L \ell_i + (\beta^R z_i^R + \gamma^R_{h_i^R}) + \gamma^P_{h_i^P} + (\beta^C z_i^C + \gamma^C_{h_i^C}),\\
            \gamma^R_{h} &\sim \mathcal{N}(0, \sigma^2_R) \hspace{4mm} \mbox{\it i.i.d.} \hspace{4mm} \forall h \in \mathcal{H}^R,\\
            \gamma^P_{h} &\sim \mathcal{N}(0, \sigma^2_P) \hspace{4mm} \mbox{\it i.i.d.} \hspace{4mm} \forall h \in \mathcal{H}^P,\\
            \gamma^C_{h} &\sim \mathcal{N}(0, \sigma^2_C) \hspace{4mm} \mbox{\it i.i.d.} \hspace{4mm} \forall h \in \mathcal{H}^C.
          \end{split}
        \end{align}
        As in (\ref{eqn:prob-sb-attempt}), we interpret $(\beta^R z_i^R + \gamma^R_{h_i^R})$ as the runner effect regularized toward a prior based on sprint speed and $(\beta^C z_i^C + \gamma^C_{h_i^C})$ as the catcher effect regularized toward a prior based on arm strength.

    \subsection{The baseball inning as a Markov decision process}

      \subsubsection{Markov chain model}

        Modeling the progression of a baseball inning is a well-trodden application of Markov chains in sports.(citation needed) Traditionally, the state space is defined by the bases occupied and the number of outs (24 non-terminal states, e.g., citation needed) or by those and the ball-strike count (288 non-terminal states, e.g., citation needed). To understand the game theory under the new pickoff limit rule, we take this a step further and include the number of disengagements in the state space.
      
        {\it State space.} The game state is given by $s = (b, c, d, o)$, where:
        \begin{align*}
          b &\in \{0, 1\}^3 \mbox{ represents which bases are occupied by runners};\\
          c &\in \{0, 1, 2, 3\} \times \{0, 1, 2\} \mbox{ represents the ball-strike count};\\
          d &\in \{0, 1, 2\} \mbox{ represents the number of disengagements already used}; and\\
          o &\in \{0, 1, 2\} \mbox{ represents the number of outs}.
        \end{align*}
        The total number of non-terminal states is $2^3 \times (4 \times 3) \times 3 \times 3 = 864$. There is one terminal state (end of inning), bringing the total number of states to 865. We use $\mathcal{S}$ to denote the set of all states, including the terminal state; we use $s_i = (b_i, c_i, d_i, o_i)$ to denote the starting state for play $i$; and we use $S_i$ to denote the random variable governing the probability distribution over $s_i$.
  
        {\it Transition probabilities.} The probability of transition to state $s'$ from starting state $s$ is given by the transition probability function $\mathbb{P}(S_{i+1} = s' \mid S_i = s)$. Because the present work focuses on the consequences of decisions surrounding the runner on first base, it is helpful to break down these transition probabilities by conditioning on the runner outcome:
        \begin{align}
          \label{eqn:transition-prob}
          \mathbb{P}(S_{i+1} = s' \mid S_i = s) = \sum_{r \in \mathcal{R}} \mathbb{P}(R_i = r \mid S_i = s) \cdot \mathbb{P}(S_{i+1} = s' \mid S_i = s, R_i = r).
        \end{align}

        To estimate $\mathbb{P}(R_i = r \mid S_i = s)$, we may use the empirical frequency of each runner outcome $r$ in each state $s$. In Section \ref{sec:single-agent-mdp} below, we will replace these empirical frequencies with the probabilities estimated in Section \ref{sec:prob-runner-outcome}.
        
        To estimate $\mathbb{P}(S_{i+1} = s' \mid S_i = s, R_i = r)$, we use conditional empirical frequencies of the ending state $s'$ given the starting state $s$ and the runner outcome $r$. However, we pool across the starting disengagements $d$ when calculating these empirical frequencies. Our assumption is that, given the runner event, the starting disengagements $d$ does not impact the transition probabilities between the reduced states $\tilde s \equiv (b, c, \cdot, o)$ and $\tilde s' \equiv (b', c', \cdot, o')$. By pooling data across different values of $d$, we mitigate small sample size issues when estimating transition probabilities from rare states, such as having a runner on third base with an 0-0 count, two disengagements and zero outs.
        
        The pooled empirical transition probability conditions on $e = d' - d$ instead of conditioning on $d$ and $d'$. Using $E_{i}$ to denote the random variable governing the probability distribution over $d_{i + 1} - d_i$,
        \begin{align*}
          Q(\tilde s, r, \tilde s', e)
            &= \hat{\mathbb{P}}\left(\tilde S_{i+1} = \tilde s',\, E_i = e \mid \tilde S_i = \tilde s,\, R_i = r\right)\\
            &= \frac
              {\sum_{i=1}^n\mathbb{I}\left\{\tilde s_i = \tilde s,\, r_i = r\right\} \cdot \mathbb{I}\left\{\tilde s_{i+1} = \tilde s',\, d_{i+1} - d_i = e\right\}}
              {\sum_{i=1}^n\mathbb{I}\left\{\tilde s_i = \tilde s,\, r_i = r\right\}}.
        \end{align*}
        Finally, we estimate the full-state transition probabilities as
        \begin{align*}
          \hat{\mathbb{P}}(S_{i+1} = (b', c', d', o') \mid S_i = (b, c, d, o), R_i = r) = Q\left\big((b, c, \cdot, o),\, r,\, (b', c', \cdot, o'),\, d' - d\right\big).
        \end{align*}

% NOTE (SP): I don't think we need the paragraph below, but I'm holding onto it for now in case we decide we do need it.
%        To convert the reduced-state estimated transition probabilities to full-state transition probabilities, we leverage the fact that the number of disengagements $d'$ in the ending state is a deterministic function of the starting disengagements $d$, the runner outcome $r$ and  the ending count $c'$:
%        \begin{align}
%          \label{eqn:disengagement-transition}
%          d' = D(d, c', r)
%            \equiv \begin{cases}
%            \hfil d & \mbox{if } r \in \{\mbox{N}\} \mbox{ and } c' \ne (0, 0)\\
%            \hfil 0 & \mbox{if } r \in \{\mbox{N}\} \mbox{ and } c' = (0, 0)\\
%            d + 1   & \mbox{if } r \in \{\mbox{P}^-\} \mbox{ and } d < 2\\
%            \hfil 0 & \mbox{if } r \in \{\mbox{P}^-\} \mbox{ and } d = 2\\
%            \hfil 0 & \mbox{if } r \in \{\mbox{P}^+,\, \mbox{S}^+,\, \mbox{S}^-\}\\
%          \end{cases}.
%        \end{align}
 
      \subsubsection{Single-agent Markov decision process}
      \label{sec:single-agent-mdp}

        The Markov decision process is an extension of the Markov chain model that allows for deterministic actions which change transition probabilities between states. In this section we allow for the runner to choose a lead distance $\ell$, and we model the pitcher's decision whether to attempt a pickoff as a stochastic process (see Section \ref{sec:prob-po-attempt}). We call this the {\it single-agent} model because only the runner has agency in this model of the game. We limit this action to states in which first base is occupied and the other bases are unoccupied, i.e. $b = (1, 0, 0)$. This limits our scope to situations in which the runner might steal second base (encompassing XX\%\footnote{TODO: fill this in} of stolen base attempts in 2023) and excludes the first-and-third scenario $b = (1, 0, 1)$, in which the first-base runner will often attempt a stolen base to lure a throw from the catcher, giving the third-base runner a chance to steal home.
     
        {\it Action space.} In states for which $b = (1, 0, 0)$, the first-base runner must choose a lead distance $\ell \in [0, 20] \equiv \mathcal{L}$, representing distance in feet from first base. For computational purposes, we discretize this action space by rounding $\ell$ to the nearest tenth of a foot, meaning that there are 201 possible actions.
  
        {\it Transition probabilities.} The probability of transition to ending state $s'$ from starting state $s$ given lead distance $\ell$ is $P(s' \mid s,\, \ell) = \mathbb{P}(S_{i+1} = s' \mid S_i = s,\, \ell_i = \ell)$. To estimate state transition probabilities that depend on lead distance, we modify (\ref{eqn:transition-prob}) by conditioning on the lead distance:
        \begin{align}
          \label{eqn:transition-prob-lead}
          P(s' \mid s,\, \ell) = \sum_{r \in \mathcal{R}} \mathbb{P}(R_i = r \mid S_i = s,\, \ell_i = \ell) \cdot \mathbb{P}(S_{i+1} = s' \mid S_i = s,\, R_i = r).
        \end{align}
        In this equation, the runner outcome depends on the lead distance, but we assume that given the runner outcome, the state transition is independent of the lead distance. The procedure for estimating $\mathbb{P}(S_{i+1} = s' \mid S_i = s,\, R_i = r)$ is described in Section \ref{sec:single-agent-mdp}. To estimate $\mathbb{P}(R_i = r \mid S_i = s,\, \ell_i = \ell)$, we use the models from Section \ref{sec:prob-runner-outcome}. Specifically, $\mathbb{P}(R_i = r \mid S_i = s,\, \ell_i = \ell,\, p_i = p)$ is given by
        \begin{align*}
          \begin{cases}
              \hfil \pi_i(\ell) \cdot \pi_i^+(\ell)                             & \mbox{if } r = \mbox{P}^+\\
              \hfil \pi_i(\ell) \cdot (1 - \pi_i^+(\ell))                       & \mbox{if } r = \mbox{P}^-\\
              \hfil (1 - \pi_i(\ell)) \cdot \psi_i \cdot \psi_i^+(\ell)         & \mbox{if } r = \mbox{S}^+\\
              \hfil (1 - \pi_i(\ell)) \cdot \psi_i \cdot (1 - \psi_i^+(\ell))   & \mbox{if } r = \mbox{S}^-\\
              \hfil (1 - \pi_i(\ell)) \cdot (1 - \psi_i)                        & \mbox{if } r = \mbox{N}\\
          \end{cases}.
        \end{align*}
        Recall that the stolen base attempt probability $\psi_i$ does not depend on $\ell$, as described in Section \ref{eqn:prob-sb-attempt}.
  
        {\it Reward function.} The reward function for transitioning from state $s$ to state $s'$ is given by the number of runs scored in the transition. Note that this reward function does not depend on the action taken by the runner, so we can write $R(s, \ell, s') = R(s, s')$. If $s = (b, c, d, o)$ and $s' = (b', c', d', o')$, then:
        \begin{align}
          \label{eqn:reward}
          R(s, s') = (g(b') + o') - (g(b) + o) - \mathbb{I}\{c' = (0, 0),\, d' = 0\},
        \end{align}
        where $g: \{0, 1\}^3 \rightarrow \{0, 1, 2, 3\}$ counts the number of runners on base. This expression for the reward function counts the total number of baserunners and outs before and after the transition and subtracts one if the transition corresponds to the end of a plate appearance. For transitions to the terminal end-of-inning state, the number of runs scored may not be uniquely determined by the starting state. To handle this case, we introduce intermediary states which track the number of runs scored on the last play, only for transitions to the terminal end-of-inning state. These intermediary states transition to the terminal end-of-inning state with probability 1.

        {\it Policy iteration.} A policy function $\pi : \mathcal{S} \rightarrow \mathcal{L}$ maps the state space to the action space, defining the lead distance to be taken by the runner in each state. The policy function $\pi$ specifies transition probabilities $P(s' \mid s,\, \ell = \pi(s))$ and induces a probability distribution over the sequence of states $\{S_t\}_{t=1}^\infty$. The value function $V_\pi : \mathcal{S} \rightarrow \mathbb{R}^+$ is the expected sum of rewards with respect to this probability distribution, which we interpret as the expected number of runs scored in the inning, starting from state $s$:
        \begin{align}
          \label{eqn:run-expectancy}
          V_\pi(s) = \mathbb{E}_\pi\left[\left.\sum_{t = 0}^\infty R(S_{t}, S_{t+1})~\right|~S_0 = s\right].
        \end{align}
        
        The relevant question for the runner is, ``What lead distance should I choose in each situation to maximize run scoring overall?'' We find the policy function $\pi^*$ which maximizes (\ref{eqn:run-expectancy}) via policy iteration (citation needed) by initializing $\hat V_0(s) = 0$ and $\hat \pi_0(s) = 0$ for all $s \in \mathcal{S}$, and then iteratively applying the updates
        \begin{align}
          \label{eqn:update-value-single-agent}
          \hat V_{i+1}(s) &= \sum_{s' \in \mathcal{S}} P(s' | s,\, \hat\pi_i(s)) [R(s, s') + \hat V_i(s')] && \forall s \in \mathcal{S}, \mbox{ and}\\
          \label{eqn:update-policy-single-agent}
          \hat\pi_{i+1}(s) &= \arg\max_{\ell \in \mathcal{L}} \sum_{s' \in \mathcal{S}} P(s' | s,\, \ell) [R(s, s') + \hat V_{i+1}(s')] && \forall s \in \mathcal{S},
        \end{align}
        until $\max_{s\in\mathcal{S}}|\pi_{i+1}(s) - \pi_i(s)|$ converges to zero.\footnote{Are we guaranteed convergence? Are we guaranteed convergence to the optimal policy?}
    
      \subsubsection{Two-agent Markov decision process}
      \label{sec:two-agent-mdp}

        In this section we extend the single-agent model by allowing the pitcher to choose whether to attempt a pickoff. We call this the {\it two-agent} model because both the runner and the pitcher have agency in this model of the game. Specifically, we model this as a two-stage game, where the runner first chooses a lead distance, and then the pitcher decides whether to attempt a pickoff after observing the runner's lead distance.

        {\it Action space.} As in the single-agent model, we limit actions to states in which only first base is occupied. The runner chooses a lead distance $\ell \in \mathcal{L}$, and then the pitcher chooses a pickoff action $p \in \{0, 1\} \equiv \mathcal{P}$, where $p = 1$ corresponds to a pickoff attempt.

        {\it Transition probabilities.} The probability of transition to ending state $s'$ from starting state $s$ given lead distance $\ell$ and pickoff decision $p$ is $P(s' \mid s,\, \ell,\, p) = \mathbb{P}(S_{i+1} = s' \mid S_i = s,\, \ell_i = \ell,\, p_i = p)$. We modify (\ref{eqn:transition-prob-lead}) by conditioning additionally on the pickoff decision:
        \begin{align}
          \label{eqn:transition-prob-lead-pickoff}
          P(s' \mid s, \ell, p) = \sum_{r \in \mathcal{R}} \mathbb{P}(R_i = r \mid S_i = s,\, \ell_i = \ell,\, p_i = p) \cdot \mathbb{P}(S_{i+1} = s' \mid S_i = s,\, R_i = r),
        \end{align}
        where $\mathbb{P}(R_i = r \mid S_i = s,\, \ell_i = \ell,\, p_i = p)$ is given by
        \begin{align*}
            \begin{cases}
                \hfil p \cdot \pi_i^+(\ell)                             & \mbox{if } r = \mbox{P}^+\\
                \hfil p \cdot (1 - \pi_i^+(\ell))                       & \mbox{if } r = \mbox{P}^-\\
                \hfil (1 - p) \cdot \psi_i \cdot \psi_i^+(\ell)         & \mbox{if } r = \mbox{S}^+\\
                \hfil (1 - p) \cdot \psi_i \cdot (1 - \psi_i^+(\ell))   & \mbox{if } r = \mbox{S}^-\\
                \hfil (1 - p) \cdot (1 - \psi_i)                        & \mbox{if } r = \mbox{N}\\
            \end{cases}.
        \end{align*}

        {\it Reward function.} The reward function (\ref{eqn:reward}) is the same for the two-agent model as for the single-agent model. In this game, however, the runner wants to maximize the reward, and the pitcher wants to minimize the reward.

        {\it Policy iteration.} In the the policy update step (\ref{eqn:update-policy-single-agent}) of policy iteration for the single-agent model, we seek the policy which maximizes the expected one-step sum of reward and resulting value, given an estimated value function. In the two-agent model, there are game theoretic considerations. Given the estimated value function $\hat V$ and the runner's chosen lead distance $\ell$, the pitcher chooses $p$ to minimize the expected one-step sum of reward and resulting value:
        \begin{align*}
          \arg\min_{p\in\mathcal{P}} \sum_{s' \in \mathcal{S}} P(s' \mid s,\, \ell,\, p) [R(s, s') + \hat V(s')].
        \end{align*}
        Knowing that this is the optimal behavior for the pitcher, the runner chooses their lead distance to maximize the resulting run value:
        \begin{align*}
          \arg\max_{\ell \in \mathcal{L}}\left\{\min_{p\in\mathcal{P}} \sum_{s' \in \mathcal{S}} P(s' \mid s,\, \ell,\, p) [R(s, s') + \hat V(s')]\right\}.
        \end{align*}

        The optimal policy for the pitcher is clear: between the choices of attempting a pickoff or not, make the choice that results in the lower run expectation. To find the optimal policy $\pi^* : \mathcal{S} \rightarrow \mathcal{L}$ for the runner, we again apply policy iteration. Starting with $\hat V_0(s) = 0$ and $\hat pi_0(s) = 0$ for all $s \in \mathcal{S}$, we iteratively apply the updates below until the policy converges.\footnote{Are we guaranteed convergence to the optimal policy?}
        
        \begin{align}
          \label{eqn:update-value-two-agent}
          \hat V_{i+1}(s) &= \min_{p \in \mathcal{P}} \sum_{s' \in \mathcal{S}} P(s' | s,\, \hat\pi_i(s)) [R(s, s') + \hat V_i(s')] && \forall s \in \mathcal{S}, \mbox{ and}\\
          \label{eqn:update-policy-two-agent}
          \hat\pi_{i+1}(s) &= \arg\max_{\ell \in \mathcal{L}} \left\{\min_{p \in \mathcal{P}}\sum_{s' \in \mathcal{S}} P(s' | s,\, \ell,\, p) [R(s, s') + \hat V_{i+1}(s')]\right\} && \forall s \in \mathcal{S}.
        \end{align}

  \section{Results}
  
    \subsection{Runner outcome probabilities given lead distance}

    First, let's look at how runners' behavior changed in MLB with the new pickoff rules introduced before the 2023 season. Figure \ref{fig:leads-overall} shows how runners' lead distances at 1st base changes from 2022 to 2023 once disengagements were limited.

    \begin{figure}[b]
      \centering
      \includegraphics[width = 0.5\textwidth]{figures/leads_overall.png}
      \caption{The curves show the distribution of the lead distances at first base in feet. The red curve shows all leads from 2022, while the blue curves show the leads from 2023 under different levels of disengagements. The vertical lines represent the mean in each situation.}
      \label{fig:leads-overall}
    \end{figure}

    In 2022, we found that the number of disengagements did not have much of an  effect on lead distances at first base. In 2023, we saw that the distribution of leads by runners under 0 disengagement situations was virtually identical to all situations from 2022. With each successive disengagement, runners' leads tended to increase by about 0.6-0.8 feet. Unsurprisingly, the rule changes did make runners slightly more aggressive when taking their leads at first base. However, runners did not increase their aggression to the level that we would have recommended.
    
     Next, we will examine the results of our runner outcome probability models. These logistic regression models help us update our transition probabilities based on the runner's lead distance, the game situation, and the players involved in a play.
      
      
        Starting with the pickoff attempt probability model, Figure \ref{fig:prob-pickoff-attempt} shows how our estimated probability changes based on lead distance and the the number of prior disengagements:

        \begin{figure}
          \label{fig:prob-pickoff-attempt}
          \centering
          \includegraphics[width = 0.5\textwidth]{figures/prob_pickoff_attempt.png}
          \caption{The y-axis shows the probability of a pickoff attempt based on the lead distance on the x-axis. The blue lines represent the number of prior disengagements in our 2023 data, and the red line compares that to data from 2022.}
        \end{figure}

        The number of disengagements has a major effect on the likelihood of a pickoff attempt. With each successive disengagement, the probability is cut by about 50\%. Even at large lead distances, the probability of a pickoff remains very low after a pitcher has performed a disengagement or two. Notably, the 2023 rule changes also caused a significant decline from 2022's frequency of pickoffs. Even when they had 0 prior disengagements, pitchers in 2023 attempted pickoffs less often than in 2022. Other factors affecting pickoff attempt probability include the count and the pitcher's tendency to attempt pickoffs. However, prior disengagements has the most significant effect.
    
      
        Next, we'll take a look at the pickoff success probability model. Figure \ref{fig:prob-pickoff-success}  shows how our estimated probability changes based on lead distance and the pitcher's skill at picking runners off:
    
        \begin{figure}
          \label{fig:prob-pickoff-success}
          \centering
          \includegraphics[width = 0.5\textwidth]{figures/prob_pickoff_success.png}
          \caption{The y-axis shows the likelihood of success on a pickoff attempt based on the lead distance on the x-axis. The blue lines represent the skill level of the pitcher at picking off runners, which was calculated in our random effects model}
        \end{figure}

        Unsurprisingly, the probability of a successful pickoff is fairly low for most lead distances. However, once leads approach 13-15 feet, the success rate of pickoffs increases quickly. The gap between the median pitcher and 90th percentile pitcher indicates that certain pitchers are uniquely talented at picking runners off at first base, but even the most talented pitchers' success rates remain low. We found that runner's skill level and the game situation do not have any significant effect on the success rate of a pickoff.

       
        Now taking a look at the stolen base success probability model, Figure \ref{fig:prob-sb-success}  shows how our estimated probability of success changes based on lead distance and the skill level of the runner. The runner skill level is derived from the combined effects of the runner's sprint speed and his ability to steal bases:

    
        \begin{figure}
          \label{fig:prob-sb-success}
          \centering
          \includegraphics[width = 0.5\textwidth]{figures/prob_sb_success.png}
          \caption{The y-axis shows the likelihood of success on a stolen base attempt based on the lead distance on the x-axis. The blue lines represent the skill of the runner in our 2023 model, and the red line compares that to a median runner from our 2022 model.}
        \end{figure}
    
        Clearly, longer leads make stolen bases much more likely to be successful. The 2023 rule changes have made stealing bases significantly easier. A 10th percentile runner in 2023 is now even more likely to be successful while stealing as a median runner was in 2022. Other factors affecting stolen base success include the pitcher and catcher skill levels and the arm strength of the catcher.


        We elected not to include lead distance in our stolen base attempt probability model. We found that the results obtained by solely modeling stolen base attempt probability based on game situation and player's involved made much more sense than if we used lead distance. For each possible combination of situation and players, we model the probability that the runner will steal and then allow the runner to choose a lead to maximize their likelihood of stealing successfully while avoiding the risk of being picked off. 

    Finally, let's also look at how our random effect models assign skill levels to runners and catchers. Figure \ref{fig:random-effect-catcher} shows how correlated our catcher skill effect is with their arm strength. 

    
        \begin{figure}
          \label{fig:random-effect-catcher}
          \centering
          \includegraphics[width = 0.5\textwidth]{figures/catcher_effect.png}
          \caption{The x-axis shows the catcher's arm strength in miles per hour, while the y-axis shows their skill at preventing stolen bases according to our random effects model. A lower y-value means a catcher is better at preventing stolen bases.}
        \end{figure}

        There's clearly a strong correlation between arm strength and catcher skill. However, there is also a decent amount of variance in catcher skill that arm strength cannot explain. This likely comes from a catcher's arm accuracy and pop time.


        Figure \ref{fig:random-effect-runner} shows how correlated our runner skill effect is with their sprint speed.

         \begin{figure}
          \label{fig:random-effect-runner}
          \centering
          \includegraphics[width = 0.5\textwidth]{figures/runner_effect.png}
          \caption{The x-axis shows the runner's sprint speed in feet per second, while the y-axis shows their skill at stealing bases according to our random effects model. A higher y-value means a runner is better at stealing bases.}
        \end{figure}

        Sprint speed definitely is a major factor in how good a runner is at stealing bases. However, it cannot explain all of the variance in skill here. Getting good jumps and skilfully avoiding tags also likely help differentiate runners in this area.

    

    \subsection{Optimal lead distance under current conditions}
    
      We use the models described in the previous section to update our transition probabilities based on the runner's lead distance. Then, for each game state, we select the lead distance that maximizes the run expectancy, update the transition probabilities again, and repeat until convergence. 
    

        Figure \ref{fig:finding-optimal-lead} illustrates how we find the optimal lead distance for a runner on first in a given situation. This shows how the run expectancy of the inning changes based on the runner's lead distance and the number of prior disengagements. This chart assumes a 0-0 count, 0 outs, an average runner, pitcher, and catcher, and nobody else on base:

        \begin{figure}
          \label{fig:finding-optimal-lead}
          \centering
          \includegraphics[width = 0.5\textwidth]{figures/finding_optimal_lead.png}
          \caption{The y-axis shows the run expectancy of each lead distance on the x-axis. The blue curves represent the number of prior disengagements in the plate appearance, and the vertical lines mark the lead that maximizes the run expectancy in each situation.}
        \end{figure}

        For each situation, we select the lead that maximizes the run expectancy for the inning. In this situation, along with most others we will examine, each successive disengagement increases the optimal lead distance by about two feet.
    
    
        Table \ref{tab:count} shows the optimal lead distance for a runner on first base based on the count and prior disengagements. This table assumes an average runner and battery, 0 outs, and no other runners on base:

        \begin{table}
          \centering
          \begin{tabular}{c|ccc}
            Count & 0 Disengagements & 1 Disengagement & 2 Disengagements\\
            \hline
            % latex table generated in R 4.3.1 by xtable 1.8-4 package
% Sun Jul 21 17:31:33 2024
 0-0 & 10.0 & 11.5 & 14.1 \\ 
  0-1 & 9.8 & 11.4 & 14.1 \\ 
  0-2 & 9.5 & 11.1 & 14.1 \\ 
  1-0 & 10.2 & 11.7 & 14.2 \\ 
  1-1 & 10.0 & 11.6 & 14.2 \\ 
  1-2 & 9.9 & 11.4 & 14.3 \\ 
  2-0 & 10.2 & 11.7 & 14.2 \\ 
  2-1 & 10.3 & 11.8 & 14.3 \\ 
  2-2 & 10.0 & 11.5 & 14.2 \\ 
  3-0 & 10.3 & 11.8 & 14.2 \\ 
  3-1 & 10.4 & 11.9 & 14.3 \\ 
  3-2 & 10.4 & 11.9 & 14.4 \\ 
  
          \end{tabular}
          \caption{Based on the count in each row, each entry shows the optimal lead distance (in feet) for a runner on 1B depending on the number of disengagements}
          \label{tab:count}
        \end{table}

        Each successive disengagement leads to an increase in recommended lead distance of approximately two feet. As pitchers approach the two disengagement threshold, the likelihood of a pickoff plummets, so runners can be more aggressive. Also, we notice that recommended leads tend to be a bit higher as the count becomes more advantageous for the batter. This is mainly because pitchers are more likely to attempt a pickoff in these counts.

    
        Table \ref{tab:runners-outs} shows the optimal lead distance for a runner on first base based on the number of outs and prior disengagements. This table assumes an average runner and battery and a 0-0 count:

        \begin{table}
          \centering
          \begin{tabular}{c|ccc}
             Outs & 0 Disengagements & 1 Disengagement & 2 Disengagements\\
            \hline
            % latex table generated in R 4.3.1 by xtable 1.8-4 package
% Sun Jul 21 17:31:33 2024
 0 & 10.0 & 11.5 & 14.1 \\ 
  1 & 10.1 & 11.6 & 14.1 \\ 
  2 & 10.8 & 12.0 & 14.4 \\ 
  
          \end{tabular}
          \caption{Based on the number of outs in each row, each entry shows the optimal lead distance (in feet) for a runner on 1B depending on the number of disengagements}
          \label{tab:runners-outs}
        \end{table}

        Again, we see that each successive disengagement leads to an increase in recommended lead distance of approximately two feet. We also see that the recommended leads for 0 and 1 out are virtually identical, and are slightly longer with 2 outs. With 2 outs, stolen bases are more likely and pickoff attempts are less likely, so runners can be more aggressive.

    
        Table \ref{tab:players} shows the optimal lead distance for a runner on first base based on the players involved in the play. The runner skill level is based on sprint speed and their ability to steal bases. The battery skill level includes the pitcher and catcher's abilities to pick runners off and throw out potential basestealers, as well as the catcher's arm strength. This table assumes a 0-0 count, 0 outs, and nobody else on base:
    
        \begin{table}
          \centering
          \begin{tabular}{cc|ccc}
            Battery Skill  & Runner Skill & 0 Disengagements & 1 Disengagements & 2 Disengagements\\
            \hline
            % latex table generated in R 4.3.1 by xtable 1.8-4 package
% Sun Jul 21 17:31:33 2024
 10th Percentile Battery & 10th Percentile Runner & 10.5 & 11.9 & 14.5 \\ 
  10th Percentile Battery & Median Runner & 11.3 & 12.5 & 14.7 \\ 
  10th Percentile Battery & 90th Percentile Runner & 12.0 & 13.0 & 14.8 \\ 
  Median Battery & 10th Percentile Runner & 9.6 & 11.3 & 14.1 \\ 
  Median Battery & Median Runner & 10.3 & 11.7 & 14.1 \\ 
  Median Battery & 90th Percentile Runner & 11.0 & 12.1 & 14.2 \\ 
  90th Percentile Battery & 10th Percentile Runner & 7.8 & 9.8 & 12.6 \\ 
  90th Percentile Battery & Median Runner & 8.2 & 10.0 & 12.7 \\ 
  90th Percentile Battery & 90th Percentile Runner & 8.8 & 10.4 & 12.8 \\ 
  
          \end{tabular}
          \caption{Based on the skill of the battery (pitcher and catcher) and the skill of the runner in each row, each entry shows the optimal lead distance for a runner on first base depending on the number of disengagements}
          \label{tab:players}
        \end{table}

        The effect of the battery seems to be more significant than the effect of the runner. Facing a stronger pitcher and catcher cause the runner's optimal lead distance to shrink by several feet. We again see consistent results that runners should increase their lead by 1.5-2.5 feet upon each successive disengagement. However, better runners don't need to increase their leads by as much, likely because their stolen base success rates will be high to begin with.

        Currently, MLB runners are not doing a great job of putting pressure on the defense by maximizing their lead distances. Table \ref{tab:actual-vs-rec} compares the average actual and recommended leads for runners at 1st base with 2nd base open. With 0 disengagements, runners' actual leads come close to our recommendation, but they could often be adding an extra foot. As disengagements increase, we recommend about a 2 foot increase, but runners only tend to increase their leads by 0.6-0.7 feet. Runners overestimate the danger of a pickoff attempt under these situations, and could be increasing their team's run expectancy by adding a few feet to their lead distances here. Runners rarely meet or exceed the lead distance we recommend, especially with 1 or 2 disengagements.

        \begin{table}
          \centering
          \begin{tabular}{c|ccc}
            Disengagements & Average Actual Lead & Average Recommended Lead & Actual Exceeds Recommendation\\
            \hline
            \begin{tabular}{c|cc|c}
 & \multicolumn{2}{c|}{Average Lead} & Actual Exceeds\\
Disengagements & Actual & Recommended & Recommendation \\
% latex table generated in R 4.5.0 by xtable 1.8-4 package
% Sat Aug  2 17:11:21 2025
  \hline
0 & 9.5 & 10.3 & 31.1\% \\ 
  1 & 10.2 & 11.7 & 12.4\% \\ 
  2 & 10.8 & 14.1 & 1.0\% \\ 
\end{tabular}

          \end{tabular}
          \caption{Each entry shows the average lead at 1st base based on the number of disengagements prior to the pitch. The recommended leads in the final column are based on our single-agent model. The final column shows the percent of pitches where the actual lead exceeds our recommendation.}
          \label{tab:actual-vs-rec}
        \end{table}


        In order to quantify the effect of optimizing baserunners' lead distances, we compared the run expectancy at the start of an inning before and after iterating through the Bellman Equation. In the 2023 season, the run expectancy of an inning was 0.5163 runs. After our policy iteration, the run expectancy jumped to 0.5277 runs. This increase of 0.0114 runs per inning may seem very small, but this difference adds up quickly. Over a 162 game season, adding 0.0114 runs per inning increases a team's run total by 16.68 runs. Using the typical valuation of \$1 Million per run, optimizing baserunners' lead distances could be worth nearly \$17 Million to a team under the current conditions. This is an incredibly valuable result that does not require much effort for a team to implement.

      \subsection{Game theoretic equilibrium}

        All of the results shown so far correspond to our single-actor model, where a runner choose a lead distance and all other outcomes are probabilistic based on their choice. Now, we expand our model to a two-actor model, where the runner first chooses a lead distance, then the pitcher elects whether to attempt a pickoff or throw a pitch. The pitcher makes whichever decision minimizes the run expectancy for the offensive team. Therefore, in order for the runner to maximize their run expectancy, they should choose a lead distance such that the pitcher is indifferent as to whether to attempt a pickoff or throw a pitch. Pitchers should always attempt a pickoff if the runner's lead extends beyond this distance, and should always throw a pitch if the lead is shorter. This is the game theoretic equilibrium.


        Table \ref{tab:count-twoagent} shows the optimal lead by count under the two-agent model:

        \begin{table}
          \centering
          \begin{tabular}{c|ccc}
            Count & 0 Disengagements & 1 Disengagement & 2 Disengagements\\
            \hline
            \begin{tabular}{r|cccc|cccc|cccc}
Prior & \multicolumn{12}{c}{Count}\\
Disengagements & 0-0 & 1-0 & 2-0 & 3-0 & 0-1 & 1-1 & 2-1 & 3-1 & 0-2 & 1-2 & 2-2 & 3-2 \\
% latex table generated in R 4.5.0 by xtable 1.8-4 package
% Sat Aug  2 20:42:12 2025
  \hline
0 & 11.1 & 11.1 & 10.9 & 10.3 & 10.7 & 10.8 & 10.8 & 10.4 & 10.1 & 10.3 & 10.1 & 10.2 \\ 
  1 & 11.6 & 11.9 & 12.0 & 11.6 & 11.2 & 11.2 & 11.9 & 11.6 & 10.6 & 10.2 & 10.3 & 11.4 \\ 
  2 & 15.7 & 15.6 & 15.4 & 14.8 & 15.7 & 15.6 & 15.4 & 15.2 & 15.7 & 15.8 & 15.5 & 15.1 \\ 
\end{tabular}

          \end{tabular}
          \caption{Based on the count in each row, each entry shows the optimal lead distance (in feet) for a runner on 1B depending on the number of disengagements for our two-agent model}
          \label{tab:count-twoagent}
        \end{table}

        Under this game-theoretic equilibrium, we see that runners should be taking even longer leads in most situations. This is likely because pickoffs are often wasteful at lower lead distances. If pitchers were to optimize their pickoff behavior, then runners would be able to get longer leads before a pickoff becomes worthwhile for the pitcher. Our two-foot rule of thumb from the single-agent model doesn't hold here. Instead, we see a recommended increase of about 0.5-1 feet after 1 disengagement, and a recommended increase of around 3.5-5 feet after the 2nd disengagement. Once a pitcher reaches 2 disengagements, it becomes extremely risky to attempt a pickoff, so runners can take advantage of that by extending their leads beyond 15 feet.

        In order to evaluate how well pitchers follow our recommendations, we looked back at all situations from the 2023 season with a runner on first base only. Our two-agent model suggests that pitchers should have attempted a pickoff in 8.3\% of these situations. In reality, they only attempted pickoffs 5.6\% of the time.

        In situations where our model suggested a pickoff, pitchers actually attempted a pickoff 10.8\% of the time in 2023.

        In situations where our model suggests throwing a pitch, pitchers actually attempted a pickoff 5.0\% of the time.

        Overall, it seems that pitchers should be attempting more pickoffs. If runners start increasing their lead distances in the way we recommend, then pitchers should respond by picking off even more often.

        Under this equilibrium of optimal pitcher and baserunner behavior, runners still hold a slight advantage compared to the status quo, but not by much. We calculated that runners could increase their team's run expectancy by about 0.0036 runs per inning by following these lead recommendations if pitchers also optimized their behavior. Over a 162 game season, this adds up to 5.2 runs. Ultimately, the current behavior of pitchers and runners is not far from the Nash Equilibrium. However, both sides could improve their team's performance by several runs by optimizing their behavior surrounding the running game.

  \section{Discussion}

  \section{Assumptions}

    \begin{itemize}
      \item NOTE (SP): Let's try including pitcher handedness in our model
      \item Successful pickoff eliminates runner with no other runner movement
      \item Number of disengagements does not affect batter outcomes
      \item Lead distance does not affect outcomes on batted balls (note: domain knowledge tells us that secondary lead is more important than the lead distance we observe, which is first pitcher movement)
      \item We are ignoring the possibility that the runner takes off for 2B on a pickoff attempt
      \item Other than pickoff attempts, no other disengagements occur
    \end{itemize}

  \bibliography{arxiv}

\end{document}
